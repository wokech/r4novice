% Options for packages loaded elsewhere
\PassOptionsToPackage{unicode}{hyperref}
\PassOptionsToPackage{hyphens}{url}
\PassOptionsToPackage{dvipsnames,svgnames,x11names}{xcolor}
%
\documentclass[
  letterpaper,
  DIV=11,
  numbers=noendperiod]{scrreprt}

\usepackage{amsmath,amssymb}
\usepackage{iftex}
\ifPDFTeX
  \usepackage[T1]{fontenc}
  \usepackage[utf8]{inputenc}
  \usepackage{textcomp} % provide euro and other symbols
\else % if luatex or xetex
  \usepackage{unicode-math}
  \defaultfontfeatures{Scale=MatchLowercase}
  \defaultfontfeatures[\rmfamily]{Ligatures=TeX,Scale=1}
\fi
\usepackage{lmodern}
\ifPDFTeX\else  
    % xetex/luatex font selection
\fi
% Use upquote if available, for straight quotes in verbatim environments
\IfFileExists{upquote.sty}{\usepackage{upquote}}{}
\IfFileExists{microtype.sty}{% use microtype if available
  \usepackage[]{microtype}
  \UseMicrotypeSet[protrusion]{basicmath} % disable protrusion for tt fonts
}{}
\makeatletter
\@ifundefined{KOMAClassName}{% if non-KOMA class
  \IfFileExists{parskip.sty}{%
    \usepackage{parskip}
  }{% else
    \setlength{\parindent}{0pt}
    \setlength{\parskip}{6pt plus 2pt minus 1pt}}
}{% if KOMA class
  \KOMAoptions{parskip=half}}
\makeatother
\usepackage{xcolor}
\setlength{\emergencystretch}{3em} % prevent overfull lines
\setcounter{secnumdepth}{5}
% Make \paragraph and \subparagraph free-standing
\ifx\paragraph\undefined\else
  \let\oldparagraph\paragraph
  \renewcommand{\paragraph}[1]{\oldparagraph{#1}\mbox{}}
\fi
\ifx\subparagraph\undefined\else
  \let\oldsubparagraph\subparagraph
  \renewcommand{\subparagraph}[1]{\oldsubparagraph{#1}\mbox{}}
\fi


\providecommand{\tightlist}{%
  \setlength{\itemsep}{0pt}\setlength{\parskip}{0pt}}\usepackage{longtable,booktabs,array}
\usepackage{calc} % for calculating minipage widths
% Correct order of tables after \paragraph or \subparagraph
\usepackage{etoolbox}
\makeatletter
\patchcmd\longtable{\par}{\if@noskipsec\mbox{}\fi\par}{}{}
\makeatother
% Allow footnotes in longtable head/foot
\IfFileExists{footnotehyper.sty}{\usepackage{footnotehyper}}{\usepackage{footnote}}
\makesavenoteenv{longtable}
\usepackage{graphicx}
\makeatletter
\def\maxwidth{\ifdim\Gin@nat@width>\linewidth\linewidth\else\Gin@nat@width\fi}
\def\maxheight{\ifdim\Gin@nat@height>\textheight\textheight\else\Gin@nat@height\fi}
\makeatother
% Scale images if necessary, so that they will not overflow the page
% margins by default, and it is still possible to overwrite the defaults
% using explicit options in \includegraphics[width, height, ...]{}
\setkeys{Gin}{width=\maxwidth,height=\maxheight,keepaspectratio}
% Set default figure placement to htbp
\makeatletter
\def\fps@figure{htbp}
\makeatother
\newlength{\cslhangindent}
\setlength{\cslhangindent}{1.5em}
\newlength{\csllabelwidth}
\setlength{\csllabelwidth}{3em}
\newlength{\cslentryspacingunit} % times entry-spacing
\setlength{\cslentryspacingunit}{\parskip}
\newenvironment{CSLReferences}[2] % #1 hanging-ident, #2 entry spacing
 {% don't indent paragraphs
  \setlength{\parindent}{0pt}
  % turn on hanging indent if param 1 is 1
  \ifodd #1
  \let\oldpar\par
  \def\par{\hangindent=\cslhangindent\oldpar}
  \fi
  % set entry spacing
  \setlength{\parskip}{#2\cslentryspacingunit}
 }%
 {}
\usepackage{calc}
\newcommand{\CSLBlock}[1]{#1\hfill\break}
\newcommand{\CSLLeftMargin}[1]{\parbox[t]{\csllabelwidth}{#1}}
\newcommand{\CSLRightInline}[1]{\parbox[t]{\linewidth - \csllabelwidth}{#1}\break}
\newcommand{\CSLIndent}[1]{\hspace{\cslhangindent}#1}

\KOMAoption{captions}{tableheading}
\makeatletter
\makeatother
\makeatletter
\@ifpackageloaded{bookmark}{}{\usepackage{bookmark}}
\makeatother
\makeatletter
\@ifpackageloaded{caption}{}{\usepackage{caption}}
\AtBeginDocument{%
\ifdefined\contentsname
  \renewcommand*\contentsname{Table of contents}
\else
  \newcommand\contentsname{Table of contents}
\fi
\ifdefined\listfigurename
  \renewcommand*\listfigurename{List of Figures}
\else
  \newcommand\listfigurename{List of Figures}
\fi
\ifdefined\listtablename
  \renewcommand*\listtablename{List of Tables}
\else
  \newcommand\listtablename{List of Tables}
\fi
\ifdefined\figurename
  \renewcommand*\figurename{Figure}
\else
  \newcommand\figurename{Figure}
\fi
\ifdefined\tablename
  \renewcommand*\tablename{Table}
\else
  \newcommand\tablename{Table}
\fi
}
\@ifpackageloaded{float}{}{\usepackage{float}}
\floatstyle{ruled}
\@ifundefined{c@chapter}{\newfloat{codelisting}{h}{lop}}{\newfloat{codelisting}{h}{lop}[chapter]}
\floatname{codelisting}{Listing}
\newcommand*\listoflistings{\listof{codelisting}{List of Listings}}
\makeatother
\makeatletter
\@ifpackageloaded{caption}{}{\usepackage{caption}}
\@ifpackageloaded{subcaption}{}{\usepackage{subcaption}}
\makeatother
\makeatletter
\@ifpackageloaded{tcolorbox}{}{\usepackage[skins,breakable]{tcolorbox}}
\makeatother
\makeatletter
\@ifundefined{shadecolor}{\definecolor{shadecolor}{rgb}{.97, .97, .97}}
\makeatother
\makeatletter
\makeatother
\makeatletter
\makeatother
\ifLuaTeX
  \usepackage{selnolig}  % disable illegal ligatures
\fi
\IfFileExists{bookmark.sty}{\usepackage{bookmark}}{\usepackage{hyperref}}
\IfFileExists{xurl.sty}{\usepackage{xurl}}{} % add URL line breaks if available
\urlstyle{same} % disable monospaced font for URLs
\hypersetup{
  pdftitle={R for Novice Programmers (1e)},
  colorlinks=true,
  linkcolor={blue},
  filecolor={Maroon},
  citecolor={Blue},
  urlcolor={Blue},
  pdfcreator={LaTeX via pandoc}}

\title{R for Novice Programmers (1e)}
\author{}
\date{}

\begin{document}
\maketitle
\ifdefined\Shaded\renewenvironment{Shaded}{\begin{tcolorbox}[interior hidden, sharp corners, frame hidden, breakable, borderline west={3pt}{0pt}{shadecolor}, enhanced, boxrule=0pt]}{\end{tcolorbox}}\fi

\renewcommand*\contentsname{Table of contents}
{
\hypersetup{linkcolor=}
\setcounter{tocdepth}{2}
\tableofcontents
}
\bookmarksetup{startatroot}

\hypertarget{welcome}{%
\chapter*{Welcome}\label{welcome}}
\addcontentsline{toc}{chapter}{Welcome}

\markboth{Welcome}{Welcome}

This is the website for ``R for Novice Programmers.'' The goal of this
book is to introduce non-programmers or those with very little
programming experience to the benefits of the R and RStudio software.
The main prerequisites for learners are basic knowledge of computer
applications and experience working with files and folders. This book
will primarily focus on the basic R concepts that are hardly emphasized,
but that may prove difficult for learners new to programming.

The online version of this book is free to use and is licensed under the
\ldots{} .

The book is written in \href{https://quarto.org/}{Quarto}.

\bookmarksetup{startatroot}

\hypertarget{introduction}{%
\chapter*{Introduction}\label{introduction}}
\addcontentsline{toc}{chapter}{Introduction}

\markboth{Introduction}{Introduction}

\hypertarget{why-did-i-write-this-book}{%
\section*{Why did I write this book?}\label{why-did-i-write-this-book}}
\addcontentsline{toc}{section}{Why did I write this book?}

\markright{Why did I write this book?}

This book is primarily intended to cater to the needs of individuals who
have a desire to learn the basics of programming. I focus on R and
RStudio because their capabilities may be relevant to a wide variety of
individuals and organizations seeking to perform basic statistical
analysis and data visualization. Personally, my skills in R and RStudio
were gained via classroom instruction, online tutorials and videos, as
well as relevant blog posts. A significant disadvantage of some of these
resources is the assumption of prior programming knowledge. To address
this, I begin the book with instructions for downloading software,
navigating the R and RStudio interfaces, and an overview of the basics
of R to decrease the cognitive load on novices.

\hypertarget{about-the-author}{%
\section*{About the author}\label{about-the-author}}
\addcontentsline{toc}{section}{About the author}

\markright{About the author}

The author of this book is a Certified Carpentries instructor and a
trainer with the Digital Research Academy. Additionally, the author
holds a PhD in Biomedical Engineering and has completed postdoctoral
fellowships in vascular biology and infectious diseases. Lastly, the
author is passionate about using R and RStudio to generate data-driven
visualizations to allow for a deeper understanding of public policy
issues.

\hypertarget{syllabus}{%
\section*{Syllabus}\label{syllabus}}
\addcontentsline{toc}{section}{Syllabus}

\markright{Syllabus}

At the end of the book, the student should be able to perform the tasks
listed in the syllabus below.

\hypertarget{tbl-syllabus-.striped-.hover-tbl-colwidths10-70-20}{}
\begin{longtable}[]{@{}
  >{\raggedright\arraybackslash}p{(\columnwidth - 4\tabcolsep) * \real{0.2394}}
  >{\raggedright\arraybackslash}p{(\columnwidth - 4\tabcolsep) * \real{0.5211}}
  >{\raggedright\arraybackslash}p{(\columnwidth - 4\tabcolsep) * \real{0.2394}}@{}}
\caption{\label{tbl-syllabus-.striped-.hover-tbl-colwidths10-70-20}Syllabus}\tabularnewline
\toprule\noalign{}
\begin{minipage}[b]{\linewidth}\raggedright
Chapter
\end{minipage} & \begin{minipage}[b]{\linewidth}\raggedright
Title
\end{minipage} & \begin{minipage}[b]{\linewidth}\raggedright
Complete? (Yes / No)
\end{minipage} \\
\midrule\noalign{}
\endfirsthead
\toprule\noalign{}
\begin{minipage}[b]{\linewidth}\raggedright
Chapter
\end{minipage} & \begin{minipage}[b]{\linewidth}\raggedright
Title
\end{minipage} & \begin{minipage}[b]{\linewidth}\raggedright
Complete? (Yes / No)
\end{minipage} \\
\midrule\noalign{}
\endhead
\bottomrule\noalign{}
\endlastfoot
& Introduction & \\
1 & Overview of R and RStudio & \\
2 & Download and Install R and RStudio & \\
3 & Navigating the R and RStudio interfaces & \\
4 & Managing your files and data & \\
5 & Importing data and saving analysis outputs & \\
6 & Basic arithmetic, arithmetic operators, and variables & \\
7 & The primary types of operators in R & \\
8 & Data Types & \\
9 & Vectors & \\
10 & Data Structures (Part I) & \\
11 & Data Structures (Part II) & \\
12 & Handling missing data & \\
& Conclusion & \\
& Appendix & \\
\end{longtable}

\hypertarget{sample-chapter-design}{%
\section*{Sample chapter design}\label{sample-chapter-design}}
\addcontentsline{toc}{section}{Sample chapter design}

\markright{Sample chapter design}

Each lesson will follow a pre-described format

\begin{enumerate}
\def\labelenumi{\roman{enumi}.}
\item
  Questions to be addressed
\item
  Learning objectives
\item
  Lesson content
\item
  Practice exercises
\item
  Lesson summary
\end{enumerate}

\hypertarget{feedback}{%
\section*{Feedback}\label{feedback}}
\addcontentsline{toc}{section}{Feedback}

\markright{Feedback}

Feedback can be provided using the following channels:

\begin{enumerate}
\def\labelenumi{\roman{enumi}.}
\item
  Email: willyokech@gmail.com
\item
  Github pull request:
\end{enumerate}

\hypertarget{summary}{%
\section*{Summary}\label{summary}}
\addcontentsline{toc}{section}{Summary}

\markright{Summary}

Overall, I believe that this book will increase both the knowledge and
confidence levels of novice programmers and allow them to perform basic
statistical analysis and simplify everyday computational tasks at home
or in their workplaces. In the next chapter, we will perform a basic
overview of the R and RStudio ecosystem.

\bookmarksetup{startatroot}

\hypertarget{sec-overview}{%
\chapter{Overview of R and RStudio}\label{sec-overview}}

\hypertarget{questions}{%
\section{Questions}\label{questions}}

\begin{itemize}
\item
  What is R? How is it related to RStudio?
\item
  What are the primary advantages of R compared to other software
  programs?
\item
  Why is R considered a powerful language for statistical computing and
  data analysis?
\item
  In what domains and industries is R widely used?
\item
  What advantages does R offer over other programming languages for data
  science tasks?
\end{itemize}

\hypertarget{learning-objectives}{%
\section{Learning Objectives}\label{learning-objectives}}

\begin{itemize}
\item
  Learn about the historical background of R and RStudio.
\item
  Understand the uses and primary advantages of R and RStudio.
\item
  Assess the various applications of R across different industries.
\end{itemize}

\hypertarget{lesson-content}{%
\section{Lesson Content}\label{lesson-content}}

\hypertarget{why-learn-r-and-rstudio}{%
\subsection{Why learn R and RStudio?}\label{why-learn-r-and-rstudio}}

Both R and Studio are free, open-source software tools that are widely
used for statistical analysis and data visualization. R is a programming
language that enables the use of code to analyze data. The primary
function of the R language is statistical analysis and this can be
performed directly in the R console. To ease the analysis process and
enhance usability, an integrated development environment (IDE), such as
RStudio is recommended. The RStudio IDE is a user-friendly interface
that allows the learner to manage multiple script files, use the
command-line terminal, easily access file inputs and outputs, and review
file/analysis history.

The R programming language software was developed by Ross Ihaka and
Robert Gentleman in 1993 (published as open-source in 1995) when they
were based at the University of Auckland. \emph{Fun fact: R represents
the first letter of the first names of the creators}. The software is
utilized by individuals working for various organizations ranging from
academic institutions and healthcare organizations to financial services
and information technology companies. In January 2024, the
\href{https://pypl.github.io/PYPL.html}{PopularitY of Programming
Language (PYPL) Index}, which is created by analyzing how often language
tutorials are searched on Google, demonstrated that R was the 6th most
popular programming language. However, in the same period, the TIOBE
index (https://www.tiobe.com/tiobe-index/) indicated that R was the 23rd
most popular language. This may result from different methodologies for
developing the rankings. RStudio is an integrated development
environment (IDE) for R that was developed by JJ Allaire. This software
contains tools that make programming in R easier.

RStudio extends R's capabilities by making it easier to import data,
write scripts, and generate visualizations and reports. The company
RStudio (now Posit since 2022) was founded in 2009 with the main goal of
``creating high quality open-source software for data scientists.''

\hypertarget{uses-of-r-and-rstudio}{%
\subsection{Uses of R and RStudio}\label{uses-of-r-and-rstudio}}

\begin{enumerate}
\def\labelenumi{\roman{enumi}.}
\item
  The R and RStudio console can be used as a complex scientific
  calculator.
\item
  The values of various data types can be assigned to variables using
  the symbol \texttt{\textless{}-} or \texttt{=}.
\item
  Built-in functions can be used to manipulate variables.
\item
  Built-in datasets can be accessed internally for analysis.
\item
  New datasets can be imported and new functions can be created for
  custom analysis.
\item
  To aid in computational analysis, there exists a large package library
  (\href{https://cran.r-project.org/}{CRAN}), as well as a lot of
  software in development to aid in computational analysis.
\end{enumerate}

\hypertarget{primary-advantages-of-r-and-rstudio}{%
\subsection{Primary advantages of R and
RStudio}\label{primary-advantages-of-r-and-rstudio}}

\begin{enumerate}
\def\labelenumi{\roman{enumi}.}
\item
  R and RStudio are free and open-source software programs which makes
  them accessible to anyone with a computer and internet connection.
  This accessibility is key in enabling learners from all socioeconomic
  levels and geographic regions to have a chance to work with
  statistical software,
\item
  A large number of user communities exists for the R/RStudio software.
  These communities (listed in the Appendix) provide learning support
  and assist with technical challenges,
\item
  Numerous freely available packages/extensions have been developed by
  the R and RStudio user communities to facilitate all forms of
  computational analysis, visualization, and publication. The
  (\href{https://cran.r-project.org/}{CRAN}) has packages that contain
  datasets as well as allow one to perform statistical analysis and data
  visualization,
\item
  R and RStudio allow for reproducible analysis where scripts and
  workflows can be shared with fellow users, and,
\item
  The R/RStudio software is cross-platform, which means that it can be
  used on Linux, Windows, and Mac operating systems.
\end{enumerate}

\hypertarget{applications-of-r-in-different-industries}{%
\subsection{Applications of R in different
industries}\label{applications-of-r-in-different-industries}}

\begin{itemize}
\item
  Bioinformatics and Healthcare: epidemiological studies, clinical trial
  analysis, and genetic data analysis.
\item
  Financial Modeling and Risk Analysis: risk management, algorithmic
  trading, trading strategies and analysis, time series analysis, and
  portfolio optimization.
\item
  Retail and Marketing: customer analytics, sales forecasting, market
  research, web analytics, and customer segmentation.
\item
  Social Sciences and Humanities: text analysis, surveys and opinion
  research, social trend analysis, and policy analysis.
\item
  Statistics and Data Analysis: hypothesis testing, data visualization,
  regression modelling, and statistical inference.
\item
  Environmental Science and Climate Change: forecasting weather
  patterns, modelling climate change, monitoring pollution levels, and
  ecological modelling.
\end{itemize}

\hypertarget{exercises}{%
\section{Exercises}\label{exercises}}

As you embark on your R/RStudio learning journey, I have listed (below)
a few questions for you to think about before we get started with the
lessons.

\begin{enumerate}
\def\labelenumi{\roman{enumi}.}
\item
  Why do you want to learn R and RStudio?
\item
  Do you currently use any other software tools for data analysis and
  visualization? What are the limitations of these tools?
\item
  What tasks do you hope to accomplish after completing this training?
\item
  Explore the various R/RStudio communities listed in the appendix and
  consider joining any one of them.
\item
  Browse some of popular R packages (on
  \href{https://cran.r-project.org/}{CRAN} or
  \href{https://r-universe.dev/search/}{R-Universe}) for different tasks
  like data visualization and statistical analysis. Pick one package
  that interests you and read about its capabilities.
\end{enumerate}

\hypertarget{summary-1}{%
\section{Summary}\label{summary-1}}

Overall, I hope you enjoyed learning about the history of R and RStudio,
and have seen the advantages of using these software tools.
Additionally, we discussed the numerous applications of R in various
industries. In the next chapter, we will look at how to download and
install both R and RStudio on your local computer.

\bookmarksetup{startatroot}

\hypertarget{sec-download-install}{%
\chapter{Download and Install R and
RStudio}\label{sec-download-install}}

\hypertarget{questions-1}{%
\section{Questions}\label{questions-1}}

\emph{In progress}

\hypertarget{learning-objectives-1}{%
\section{Learning Objectives}\label{learning-objectives-1}}

\emph{In progress}

\hypertarget{lesson-content-1}{%
\section{Lesson Content}\label{lesson-content-1}}

\emph{In progress}

\hypertarget{exercises-1}{%
\section{Exercises}\label{exercises-1}}

\emph{In progress}

\hypertarget{summary-2}{%
\section{Summary}\label{summary-2}}

\emph{In progress}

\bookmarksetup{startatroot}

\hypertarget{sec-navigating}{%
\chapter{Navigating the R and RStudio interfaces}\label{sec-navigating}}

\hypertarget{questions-2}{%
\section{Questions}\label{questions-2}}

\emph{In progress}

\hypertarget{learning-objectives-2}{%
\section{Learning Objectives}\label{learning-objectives-2}}

\emph{In progress}

\hypertarget{lesson-content-2}{%
\section{Lesson Content}\label{lesson-content-2}}

\emph{In progress}

\hypertarget{exercises-2}{%
\section{Exercises}\label{exercises-2}}

\emph{In progress}

\hypertarget{summary-3}{%
\section{Summary}\label{summary-3}}

\emph{In progress}

\bookmarksetup{startatroot}

\hypertarget{sec-managing}{%
\chapter{Managing your files and data}\label{sec-managing}}

\hypertarget{questions-3}{%
\section{Questions}\label{questions-3}}

\emph{In progress}

\hypertarget{learning-objectives-3}{%
\section{Learning Objectives}\label{learning-objectives-3}}

\emph{In progress}

\hypertarget{lesson-content-3}{%
\section{Lesson Content}\label{lesson-content-3}}

\emph{In progress}

\hypertarget{exercises-3}{%
\section{Exercises}\label{exercises-3}}

\emph{In progress}

\hypertarget{summary-4}{%
\section{Summary}\label{summary-4}}

\emph{In progress}

\bookmarksetup{startatroot}

\hypertarget{sec-import-save}{%
\chapter{Importing data and saving analysis
outputs}\label{sec-import-save}}

\hypertarget{questions-4}{%
\section{Questions}\label{questions-4}}

\emph{In progress}

\hypertarget{learning-objectives-4}{%
\section{Learning Objectives}\label{learning-objectives-4}}

\emph{In progress}

\hypertarget{lesson-content-4}{%
\section{Lesson Content}\label{lesson-content-4}}

\emph{In progress}

\hypertarget{exercises-4}{%
\section{Exercises}\label{exercises-4}}

\emph{In progress}

\hypertarget{summary-5}{%
\section{Summary}\label{summary-5}}

\emph{In progress}

\bookmarksetup{startatroot}

\hypertarget{sec-arithmetic-variables}{%
\chapter{Basic arithmetic, arithmetic operators, and
variables}\label{sec-arithmetic-variables}}

\hypertarget{questions-5}{%
\section{Questions}\label{questions-5}}

\emph{In progress}

\hypertarget{learning-objectives-5}{%
\section{Learning Objectives}\label{learning-objectives-5}}

\emph{In progress}

\hypertarget{lesson-content-5}{%
\section{Lesson Content}\label{lesson-content-5}}

\emph{In progress}

\hypertarget{exercises-5}{%
\section{Exercises}\label{exercises-5}}

\emph{In progress}

\hypertarget{summary-6}{%
\section{Summary}\label{summary-6}}

\emph{In progress}

\bookmarksetup{startatroot}

\hypertarget{sec-operators}{%
\chapter{The primary types of operators in R}\label{sec-operators}}

\hypertarget{questions-6}{%
\section{Questions}\label{questions-6}}

\emph{In progress}

\hypertarget{learning-objectives-6}{%
\section{Learning Objectives}\label{learning-objectives-6}}

\emph{In progress}

\hypertarget{lesson-content-6}{%
\section{Lesson Content}\label{lesson-content-6}}

\emph{In progress}

\hypertarget{exercises-6}{%
\section{Exercises}\label{exercises-6}}

\emph{In progress}

\hypertarget{summary-7}{%
\section{Summary}\label{summary-7}}

\emph{In progress}

\bookmarksetup{startatroot}

\hypertarget{sec-data-types}{%
\chapter{Data Types}\label{sec-data-types}}

\hypertarget{questions-7}{%
\section{Questions}\label{questions-7}}

\emph{In progress}

\hypertarget{learning-objectives-7}{%
\section{Learning Objectives}\label{learning-objectives-7}}

\emph{In progress}

\hypertarget{lesson-content-7}{%
\section{Lesson Content}\label{lesson-content-7}}

\emph{In progress}

\hypertarget{exercises-7}{%
\section{Exercises}\label{exercises-7}}

\emph{In progress}

\hypertarget{summary-8}{%
\section{Summary}\label{summary-8}}

\emph{In progress}

\bookmarksetup{startatroot}

\hypertarget{sec-vectors}{%
\chapter{Vectors}\label{sec-vectors}}

\hypertarget{questions-8}{%
\section{Questions}\label{questions-8}}

\emph{In progress}

\hypertarget{learning-objectives-8}{%
\section{Learning Objectives}\label{learning-objectives-8}}

\emph{In progress}

\hypertarget{lesson-content-8}{%
\section{Lesson Content}\label{lesson-content-8}}

\emph{In progress}

\hypertarget{exercises-8}{%
\section{Exercises}\label{exercises-8}}

\emph{In progress}

\hypertarget{summary-9}{%
\section{Summary}\label{summary-9}}

\emph{In progress}

\bookmarksetup{startatroot}

\hypertarget{sec-data-structure-1}{%
\chapter{Data Structures (Part I)}\label{sec-data-structure-1}}

\hypertarget{questions-9}{%
\section{Questions}\label{questions-9}}

\emph{In progress}

\hypertarget{learning-objectives-9}{%
\section{Learning Objectives}\label{learning-objectives-9}}

\emph{In progress}

\hypertarget{lesson-content-9}{%
\section{Lesson Content}\label{lesson-content-9}}

\emph{In progress}

\hypertarget{exercises-9}{%
\section{Exercises}\label{exercises-9}}

\emph{In progress}

\hypertarget{summary-10}{%
\section{Summary}\label{summary-10}}

\emph{In progress}

\bookmarksetup{startatroot}

\hypertarget{sec-data-structure-2}{%
\chapter{Data Structures (Part II)}\label{sec-data-structure-2}}

\hypertarget{questions-10}{%
\section{Questions}\label{questions-10}}

\emph{In progress}

\hypertarget{learning-objectives-10}{%
\section{Learning Objectives}\label{learning-objectives-10}}

\emph{In progress}

\hypertarget{lesson-content-10}{%
\section{Lesson Content}\label{lesson-content-10}}

\emph{In progress}

\hypertarget{exercises-10}{%
\section{Exercises}\label{exercises-10}}

\emph{In progress}

\hypertarget{summary-11}{%
\section{Summary}\label{summary-11}}

\emph{In progress}

\bookmarksetup{startatroot}

\hypertarget{sec-missing}{%
\chapter{Handling missing data}\label{sec-missing}}

\hypertarget{questions-11}{%
\section{Questions}\label{questions-11}}

\emph{In progress}

\hypertarget{learning-objectives-11}{%
\section{Learning Objectives}\label{learning-objectives-11}}

\emph{In progress}

\hypertarget{lesson-content-11}{%
\section{Lesson Content}\label{lesson-content-11}}

\emph{In progress}

\hypertarget{exercises-11}{%
\section{Exercises}\label{exercises-11}}

\emph{In progress}

\hypertarget{summary-12}{%
\section{Summary}\label{summary-12}}

\emph{In progress}

\bookmarksetup{startatroot}

\hypertarget{conclusion}{%
\chapter*{Conclusion}\label{conclusion}}
\addcontentsline{toc}{chapter}{Conclusion}

\markboth{Conclusion}{Conclusion}

\textbf{We have finally come to the end of the ``R for Novice
Programmers'' book. Congratulations on completing this learning
journey!}

The overall goal of this book was to introduce non-programmers or those
with very little programming experience to R and RStudio and their use
in basic statistical programming. It is hoped that the learners gained
new skills and discovered how this software could be used to simplify
everyday tasks in their workplaces. The learner should now be able to
understand all the topics listed below based on the previously described
learning objectives.

\hypertarget{tbl-syllabus-.striped-.hover-tbl-colwidths10-70-20}{}
\begin{longtable}[]{@{}
  >{\raggedright\arraybackslash}p{(\columnwidth - 4\tabcolsep) * \real{0.2500}}
  >{\raggedright\arraybackslash}p{(\columnwidth - 4\tabcolsep) * \real{0.5000}}
  >{\raggedright\arraybackslash}p{(\columnwidth - 4\tabcolsep) * \real{0.2500}}@{}}
\caption{\label{tbl-syllabus-.striped-.hover-tbl-colwidths10-70-20}Syllabus}\tabularnewline
\toprule\noalign{}
\begin{minipage}[b]{\linewidth}\raggedright
Chapter
\end{minipage} & \begin{minipage}[b]{\linewidth}\raggedright
Title
\end{minipage} & \begin{minipage}[b]{\linewidth}\raggedright
Complete? (Yes / No)
\end{minipage} \\
\midrule\noalign{}
\endfirsthead
\toprule\noalign{}
\begin{minipage}[b]{\linewidth}\raggedright
Chapter
\end{minipage} & \begin{minipage}[b]{\linewidth}\raggedright
Title
\end{minipage} & \begin{minipage}[b]{\linewidth}\raggedright
Complete? (Yes / No)
\end{minipage} \\
\midrule\noalign{}
\endhead
\bottomrule\noalign{}
\endlastfoot
& Introduction & \\
1 & Overview of R and RStudio & \\
2 & Download and Install R and RStudio & \\
3 & Navigating the R and RStudio interfaces & \\
4 & Managing your files and data & \\
5 & Importing data and saving analysis outputs & \\
6 & Basic arithmetic, arithmetic operators, and variables & \\
7 & The primary types of operators in R & \\
8 & Data Types & \\
9 & Vectors & \\
10 & Data Structures (Part I) & \\
11 & Data Structures (Part II) & \\
12 & Handling missing data & \\
& Conclusion & \\
& Appendix & \\
\end{longtable}

Now that the learner has mastered the basics, it is time to transition
to specific topics such as data wrangling/manipulation, data
visualization, and programming. Therefore, I would encourage the learner
to review some of the links I have provided in the Appendix and look out
for more books in the future which will cover:

\begin{enumerate}
\def\labelenumi{\roman{enumi}.}
\item
  Data visualization
\item
  Tidyverse
\item
  Programming: Functions, Loops, and Control Statements
\item
  Data Science
\item
  Statistics
\item
  Machine Learning and Artificial Intelligence
\item
  Publishing with R Markdown and/or Quarto
\item
  R Shiny
\end{enumerate}

\textbf{Thanks for your participation and for completing this book!}

\bookmarksetup{startatroot}

\hypertarget{appendix}{%
\chapter*{Appendix}\label{appendix}}
\addcontentsline{toc}{chapter}{Appendix}

\markboth{Appendix}{Appendix}

To further your R learning journey, I would recommend a review of the
following freely available resources. This list will constantly be
updated as new material becomes available.

\hypertarget{online-repositories-with-rrstudio-related-links}{%
\section*{1. Online repositories with R/RStudio-related
links}\label{online-repositories-with-rrstudio-related-links}}
\addcontentsline{toc}{section}{1. Online repositories with
R/RStudio-related links}

\markright{1. Online repositories with R/RStudio-related links}

\begin{itemize}
\tightlist
\item
  \href{https://www.bigbookofr.com/}{Big Book of R}
\item
  \href{https://committedtotape.shinyapps.io/freeR/}{Free R Reading
  Material}
\item
  \href{https://github.com/qinwf/awesome-R}{Awesome R 1}
\item
  \href{https://github.com/uhub/awesome-r}{Awesome R 2}
\item
  \href{https://github.com/ktaranov/AwesomeR}{Awesome R 3}
\item
  \href{https://github.com/iamericfletcher/awesome-r-learning-resources}{Awesome
  R 4}
\item
  \href{https://www.resourcesdatabase.com/}{resouRces: Database of
  Resources to Learn \& Teach R}
\item
  \href{https://r-universe.dev/search/}{R-universe}
\item
  \href{https://rviews.rstudio.com/2021/11/04/bookdown-org/}{R Community
  Public Library}
\end{itemize}

\hypertarget{r-communities}{%
\section*{2. R Communities}\label{r-communities}}
\addcontentsline{toc}{section}{2. R Communities}

\markright{2. R Communities}

\begin{itemize}
\tightlist
\item
  \href{https://community.rstudio.com/}{Posit (formerly RStudio)
  Community}
\item
  \href{https://stackoverflow.com/questions/tagged/r}{Stack Overflow R}
\item
  \href{https://stackoverflow.com/questions/tagged/rstudio}{Stack
  Overflow RStudio}
\item
  \href{https://stats.stackexchange.com/?tags=r}{Cross Validated}
\item
  \href{https://www.r-bloggers.com/}{R-bloggers}
\item
  \href{https://rfordatasci.com/}{R for Data Science Slack Group}
\item
  \href{https://www.linkedin.com/company/r-user-community/}{R User
  Community}
\item
  \href{https://rladies.org/}{R-Ladies Global}
\item
  \href{https://satrdays.org/}{satRday}
\item
  \href{https://ropensci.org/community/}{rOpenSci}
\item
  \textbf{Twitter (now X) -- \#rstats and \#rstudio}
\item
  \textbf{Reddit -- r/rstats and r/RStudio}
\end{itemize}

\hypertarget{conferences-and-meetups}{%
\section*{3. Conferences and Meetups}\label{conferences-and-meetups}}
\addcontentsline{toc}{section}{3. Conferences and Meetups}

\markright{3. Conferences and Meetups}

\begin{itemize}
\tightlist
\item
  \href{https://www.r-project.org/conferences/}{R Foundation Conferences
  (useR! and DSC)}
\item
  \href{https://rstats.ai/}{R Conference}
\item
  \href{https://posit.co/conference/}{Posit Conference}
\item
  \href{https://www.meetup.com/pro/r-user-groups/}{R User Group Meetups}
\end{itemize}

\hypertarget{cheatsheets}{%
\section*{4. Cheatsheets}\label{cheatsheets}}
\addcontentsline{toc}{section}{4. Cheatsheets}

\markright{4. Cheatsheets}

\begin{itemize}
\tightlist
\item
  \href{https://rviews.rstudio.com/2021/03/10/rstudio-open-source-resorurces/}{R
  Views Cheatsheets}
\item
  \href{https://posit.co/resources/cheatsheets/}{Posit Cheatsheets}
\item
  \href{https://www.kdnuggets.com/2017/09/essential-data-science-machine-learning-deep-learning-cheat-sheets.html}{30
  Essential Data Science, Machine Learning \& Deep Learning Cheat
  Sheets}
\end{itemize}

\hypertarget{tutorials}{%
\section*{5. Tutorials}\label{tutorials}}
\addcontentsline{toc}{section}{5. Tutorials}

\markright{5. Tutorials}

\begin{itemize}
\tightlist
\item
  \href{https://swirlstats.com/}{Swirl: Learn or Teach R, in R}
\item
  \href{https://education.rstudio.com/learn/}{RStudio Education
  Learning}
\item
  \href{https://education.rstudio.com/teach/}{RStudio Education
  Teaching}
\item
  \href{https://hbiostat.org/rflow/}{R Workflow}
\item
  \href{https://www.rscreencasts.com/}{R Screencasts}
\item
  \href{https://www.statmethods.net/}{Quick-R}
\item
  \href{https://skranz.github.io/RTutor/}{RTutor: Interactive R Problem
  Sets}
\end{itemize}

\hypertarget{newsletters-and-blogs}{%
\section*{6. Newsletters and Blogs}\label{newsletters-and-blogs}}
\addcontentsline{toc}{section}{6. Newsletters and Blogs}

\markright{6. Newsletters and Blogs}

\begin{itemize}
\tightlist
\item
  \href{https://rweekly.org/}{R Weekly}
\item
  \href{https://dataelixir.com/}{Data Elixir}
\item
  \href{https://blog.revolutionanalytics.com/}{Revolution Analytics
  Blog}
\end{itemize}

\hypertarget{data-visualization}{%
\section*{7. Data Visualization}\label{data-visualization}}
\addcontentsline{toc}{section}{7. Data Visualization}

\markright{7. Data Visualization}

\begin{itemize}
\tightlist
\item
  \href{https://r-graph-gallery.com/}{The R Graph Gallery}
\item
  \href{https://r-charts.com/}{R Charts}
\item
  \href{https://exts.ggplot2.tidyverse.org/}{ggplot2 extensions}
\item
  \href{https://tellingstorieswithdata.com/}{Telling Stories with Data}
\item
  \href{https://r-graphics.org/}{R Graphics Cookbook}
\item
  \href{https://clauswilke.com/dataviz/}{Fundamentals of Data
  Visualization}
\item
  \href{https://ggplot2-book.org/}{ggplot2: Elegant Graphics for Data
  Analysis}
\item
  \href{https://socviz.co/}{Data Visualization: A practical
  introduction}
\item
  \href{https://rkabacoff.github.io/datavis/}{Modern Data Visualization
  with R}
\end{itemize}

\hypertarget{data-science}{%
\section*{8. Data Science}\label{data-science}}
\addcontentsline{toc}{section}{8. Data Science}

\markright{8. Data Science}

\begin{itemize}
\tightlist
\item
  \href{https://r4ds.hadley.nz/}{R for Data Science}
\item
  \href{https://www.data8.org/}{Data 8: The Foundations of Data Science}
\item
  \href{https://sml201.github.io/}{Introduction to Data Science}
\item
  \href{https://github.com/hadley/stats337}{Readings in Applied Data
  Science}
\item
  \href{https://github.com/AllenDowney/ElementsOfDataScience}{Elements
  of Data Science}
\item
  \href{https://bookdown.org/rdpeng/artofdatascience/}{The Art of Data
  Science}
\item
  \href{https://github.com/ml874/Data-Science-Cheatsheet}{Data Science
  Cheatsheet}
\item
  \href{https://florian-huber.github.io/data_science_course/book/cover.html}{Introduction
  to Data Science (for not-yet scientists)}
\item
  \href{https://data.org/}{data.org}
\item
  \href{https://databasic.io/en/}{Data Basic}
\item
  \href{https://datasciencebox.org/}{Data Science in a Box}
\item
  \href{http://dataliteracy.rbind.io/}{Data Literacy}
\end{itemize}

\hypertarget{statistics}{%
\section*{9. Statistics}\label{statistics}}
\addcontentsline{toc}{section}{9. Statistics}

\markright{9. Statistics}

\begin{itemize}
\tightlist
\item
  \href{https://www.statlearning.com/}{An Introduction to Statistical
  Learning with R or Python}
\item
  \href{https://modernstatisticswithr.com/}{Modern Statistics with R}
\item
  \href{https://openintro-ims.netlify.app/index.html}{Introduction to
  Modern Statistics}
\item
  \href{https://www.fharrell.com/}{Statistical Thinking}
\item
  \href{https://openintro-ims.netlify.app/index.html}{Introduction to
  Modern Statistics}
\item
  \href{https://hbiostat.org/bbr/}{Biostatistics for Biomedical
  Research}
\item
  \href{https://sta-212-f19.lucymcgowan.com/}{STA 212: Statistical
  Models}
\item
  \href{https://sta-312-f20.netlify.app/}{STA 312: Linear Models}
\item
  \href{https://sta-363-s20.lucymcgowan.com/}{STA 363: Statistical
  Learning}
\item
  \href{https://probability4datascience.com/}{Introduction to
  Probability for Data Science}
\item
  \href{https://cal-poly-advanced-r.github.io/STAT-431/}{Statistics 431:
  Advanced Statistical Computing with R}
\item
  \href{https://tinystats.github.io/teacups-giraffes-and-statistics/index.html}{Statistics
  and R}
\item
  \href{https://themockup.blog/posts/2018-12-10-a-gentle-guide-to-tidy-statistics-in-r/}{Tidy
  Statistics in R} -
  \href{https://discourse.datamethods.org/}{Datamethods}
\item
  \href{https://www.stephaniehicks.com/jhustatcomputing2022/schedule}{Statistical
  Computing}
\end{itemize}

\hypertarget{datasets}{%
\section*{10. Datasets}\label{datasets}}
\addcontentsline{toc}{section}{10. Datasets}

\markright{10. Datasets}

\begin{itemize}
\tightlist
\item
  \href{https://github.com/awesomedata/awesome-public-datasets}{Awesome
  Public Datasets}
\item
  \href{https://dataverse.org/}{Dataverse Project}
\item
  \href{https://datacommons.org/}{Data Commons}
\item
  \href{https://africaopendata.org/}{Open Africa Data}
\item
  \href{https://dataafrica.io/}{Data Africa}
\item
  \href{https://ourworldindata.org/}{Our World in Data}
\item
  \href{https://hillcrestadvisory.com/2019/01/20/california-data-sources/}{California
  Data Sources}
\item
  \href{https://datasf.org/opendata/}{DataSF}
\item
  \href{https://opendata.cityofnewyork.us/}{NYC Open Data}
\item
  \href{https://data.humdata.org/}{The Humanitarian Data Exchange}
\item
  \href{https://globaldatabarometer.org/}{Global Data Barometer}
\item
  \href{https://msropendata.com/}{Microsoft Research Open Data}
\item
  \href{https://www.kaggle.com/datasets}{Kaggle Datasets}
\item
  \href{https://datasetsearch.research.google.com/}{Google Research
  Datasets}
\item
  \href{https://nhs-r-community.github.io/NHSRdatasets/}{NHS R-community
  Datasets}
\item
  \href{https://www.healthdata.org/}{IHME}
\item
  \href{https://healthdatascience.substack.com/p/best-public-datasets-for-public-health-225}{Best
  Public Datasets for Health Data Science Projects}
\item
  \href{https://data.worldbank.org/}{World Bank Open Data}
\end{itemize}

\hypertarget{tidyverse}{%
\section*{11. Tidyverse}\label{tidyverse}}
\addcontentsline{toc}{section}{11. Tidyverse}

\markright{11. Tidyverse}

\begin{itemize}
\tightlist
\item
  \href{https://dominicroye.github.io/en/2020/a-very-short-introduction-to-tidyverse/}{A
  very short introduction to Tidyverse}
\item
  \href{https://github.com/nuitrcs/r-tidyverse}{Tidyverse Workshop
  Series}
\item
  \href{https://jhudatascience.org/tidyversecourse/}{Tidyverse Skills
  for Data Science}
\item
  \href{https://tomaztsql.wordpress.com/2022/07/14/eight-r-tidyverse-tips-for-everyday-data-engineering/}{Eight
  R Tidyverse tips for everyday data engineering}
\item
  \href{https://oliviergimenez.github.io/tidyverse-tips/}{Tidyverse
  Tips}
\item
  \href{https://pmacdasci.github.io/r-intro-tidyverse/}{An Introduction
  to R through the tidyverse}
\item
  \href{http://modern-rstats.eu/}{Modern R with the tidyverse}
\item
  \href{https://thinkr.fr/c-est-quoi-le-tidyverse/\#Mettre_en_forme_ses_donnees_avec_la_package_tidyr}{C'est
  quoi, le tidyverse?}
\item
  \href{https://www.rebeccabarter.com/blog/2019-08-05_base_r_to_tidyverse/}{Transitioning
  into the tidyverse (part 1)}
\item
  \href{https://www.rebeccabarter.com/blog/2019-08-05_base_r_to_tidyverse_pt2/}{Transitioning
  into the tidyverse (part 2)}
\end{itemize}

\hypertarget{programming-functions-loops-and-control-statements}{%
\section*{12. Programming: Functions, Loops, and Control
Statements}\label{programming-functions-loops-and-control-statements}}
\addcontentsline{toc}{section}{12. Programming: Functions, Loops, and
Control Statements}

\markright{12. Programming: Functions, Loops, and Control Statements}

\begin{itemize}
\tightlist
\item
  \href{https://cran.r-project.org/doc/manuals/r-release/R-lang.html\#Control-structures}{Control
  Structures}
\item
  \href{https://www.rdocumentation.org/packages/base/versions/3.6.2/topics/apply}{Apply
  Functions}
\item
  \href{https://modern-rstats.eu/defining-your-own-functions.html}{Defining
  your own functions}
\item
  \href{https://modern-rstats.eu/functional-programming.html}{Functional
  programming 1}
\item
  \href{http://adv-r.had.co.nz/Functional-programming.html}{Functional
  programming 2}
\item
  \href{https://www.stephaniehicks.com/jhustatprogramming2022/schedule}{Statistical
  Programming Paradigms and Workflows}
\item
  \href{https://jjallaire.github.io/hopr/}{Hands-On Programming with R}
\item
  \href{https://jennybc.github.io/purrr-tutorial/index.html}{purr
  tutorial}
\item
  \href{https://appsilon.com/functional-programming-in-r-part-1/}{Unlocking
  the Power of Functional Programming in R (Part 1)}
\item
  \href{https://appsilon.com/functional-programming-in-r-part-2/}{Unlocking
  the Power of Functional Programming in R (Part 2)}
\end{itemize}

\hypertarget{machine-learning-and-artificial-intelligence}{%
\section*{13. Machine Learning and Artificial
Intelligence}\label{machine-learning-and-artificial-intelligence}}
\addcontentsline{toc}{section}{13. Machine Learning and Artificial
Intelligence}

\markright{13. Machine Learning and Artificial Intelligence}

\begin{itemize}
\tightlist
\item
  \href{https://bradleyboehmke.github.io/HOML/}{Hands-On Machine
  Learning with R}
\item
  \href{https://rpubs.com/eR_ic/exploRe}{Create machine learning models:
  An R version}
\item
  \href{https://advanced-ds-in-r.netlify.app/posts/2021-03-31-imllocal/}{Interpretable
  Machine Learning}
\item
  \href{https://smltar.com/}{Supervised Machine Learning for Text
  Analysis in R}
\item
  \href{https://blogs.rstudio.com/ai/}{Posit AI Blog}
\item
  \href{https://www.tmwr.org/}{Tidy Modeling with R}
\item
  \href{https://www.tidymodels.org/}{Tidymodels}
\item
  \href{https://rviews.rstudio.com/2019/06/19/a-gentle-intro-to-tidymodels/}{A
  Gentle Introduction to tidymodels}
\item
  \href{https://rviews.rstudio.com/2020/04/21/the-case-for-tidymodels/}{The
  Case for tidymodels}
\item
  \href{https://emilhvitfeldt.github.io/ISLR-tidymodels-labs/index.html}{ISLR
  tidymodels labs}
\item
  \href{https://education.rstudio.com/blog/2020/02/conf20-intro-ml/}{Introduction
  to Machine Learning with the Tidyverse}
\item
  \href{https://tidypredict.tidymodels.org/}{tidypredict}
\end{itemize}

\hypertarget{publishing}{%
\section*{14. Publishing}\label{publishing}}
\addcontentsline{toc}{section}{14. Publishing}

\markright{14. Publishing}

\begin{itemize}
\tightlist
\item
  \href{https://quarto.org/}{Quarto}
\item
  \href{https://rmarkdown.rstudio.com/}{R Markdown}
\item
  \href{https://github.com/mcanouil/awesome-quarto}{Awesome Quarto}
\end{itemize}

\hypertarget{r-shiny}{%
\section*{15. R Shiny}\label{r-shiny}}
\addcontentsline{toc}{section}{15. R Shiny}

\markright{15. R Shiny}

\begin{itemize}
\tightlist
\item
  \href{https://shiny.rstudio.com/tutorial/}{Shiny Learning Resources}
\item
  \href{https://shiny.rstudio.com/tutorial/written-tutorial/lesson1/}{Shiny
  Tutorials}
\item
  \href{https://mastering-shiny.org/}{Mastering Shiny}
\item
  \href{https://rstudio-education.github.io/shiny-course/}{Building Web
  Applications WITH SHINY}
\item
  \href{https://laderast.github.io/gradual_shiny/}{a gRadual
  intRoduction to Shiny}
\item
  \href{https://engineering-shiny.org/}{Engineering Production-Grade
  Shiny Apps}
\item
  \href{https://rstudio.github.io/shinyuieditor/}{ShinyUI Editor}
\item
  \href{https://shinylive.io/r/examples/}{Shiny Examples}
\item
  \href{https://ucsb-meds.github.io/shiny-workshop/\#1}{An Intro to
  Shiny}
\item
  \href{https://shiny.posit.co/r/gallery/}{Shiny Gallery}
\item
  \href{https://rstudio.github.io/shinydashboard/structure.html}{shinydashboard}
\item
  \href{https://shiny.posit.co/r/articles/build/layout-guide/}{Application
  layout guide}
\end{itemize}

\bookmarksetup{startatroot}

\hypertarget{references}{%
\chapter*{References}\label{references}}
\addcontentsline{toc}{chapter}{References}

\markboth{References}{References}

\hypertarget{refs}{}
\begin{CSLReferences}{0}{0}
\end{CSLReferences}



\end{document}
